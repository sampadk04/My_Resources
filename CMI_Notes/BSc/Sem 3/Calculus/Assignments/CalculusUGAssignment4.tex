 \documentclass[11pt]{amsart}


\usepackage[all]{xy}
\usepackage{graphics}
\usepackage{enumitem}
\usepackage{epsfig}
\usepackage{amsmath}
\usepackage{amscd}
\usepackage{tikz-cd}
\usepackage{verbatim}
\usepackage{pdfpages}
%\usepackage{showkeys}
\usepackage{amsfonts,latexsym,amssymb}
\usepackage{parskip}
\usepackage{MnSymbol}
\usepackage{hyperref}

\usepackage{mdwlist}


%%%%%%%%%%%%%%%%%%%%%%%%%%%%%%%%%%%%%%%%%%%%%%%%%%%%%%%%


\newcommand{\cA}{{\mathcal{A}}}   \newcommand{\cB}{{\mathcal{B}}}
\newcommand{\cC}{{\mathcal{C}}}   \newcommand{\cD}{{\mathcal{D}}}
\newcommand{\cE}{{\mathcal{E}}}   \newcommand{\cF}{{\mathcal{F}}}
\newcommand{\cG}{{\mathcal{G}}}   \newcommand{\cH}{{\mathcal{H}}}
\newcommand{\cI}{{\mathcal{I}}}   \newcommand{\cJ}{{\mathcal{J}}}
\newcommand{\cK}{{\mathcal{K}}}   \newcommand{\cL}{{\mathcal{L}}}
\newcommand{\cM}{{\mathcal{M}}}   \newcommand{\cN}{{\mathcal{N}}}
\newcommand{\cO}{{\mathcal{O}}}   \newcommand{\cP}{{\mathcal{P}}}
\newcommand{\cQ}{{\mathcal{Q}}}   \newcommand{\cR}{{\mathcal{R}}}
\newcommand{\cS}{{\mathcal{S}}}   \newcommand{\cT}{{\mathcal{T}}}
\newcommand{\cU}{{\mathcal{U}}}   \newcommand{\cV}{{\mathcal{V}}}
\newcommand{\cW}{{\mathcal{W}}}   \newcommand{\cX}{{\mathcal{X}}}
\newcommand{\cY}{{\mathcal{Y}}}   \newcommand{\cZ}{{\mathcal{Z}}}

\newcommand{\hcP}{\hat{\mathcal{P}}}
\newcommand{\hcQ}{\hat{\mathcal{Q}}}
\newcommand{\hcR}{\hat{\mathcal{R}}}
\newcommand{\hcL}{\hat{\mathcal{L}}}
\newcommand{\hcM}{\hat{\mathcal{M}}} \newcommand{\hphi}{\hat{\phi}}
\newcommand{\bbk}{\mathbb{k}}   \newcommand{\bfv}{\mathbf{v}}
\newcommand{\bfnu}{\mathbf{nu}}  \newcommand{\hXX}{\hat{\mathbb{X}}}
\newcommand{\bJ}{\mathbf{J}}

\newcommand{\hD}{{\hat{D}}}   \newcommand{\hE}{{\hat{E}}}
\newcommand{\hF}{{\hat{F}}}   \newcommand{\hH}{{\hat{H}}}
\newcommand{\hY}{{\hat{Y}}}   \newcommand{\hP}{{\hat{P}}}
\newcommand{\hT}{{\hat{T}}}   \newcommand{\hQ}{{\hat{Q}}}
\newcommand{\hq}{{\hat{q}}}
\newcommand{\hr}{{\hat{r}}}
\newcommand{\hu}{{\hat{u}}}
\newcommand{\hv}{{\hat{v}}}
\newcommand{\hf}{{\hat{f}}}
\newcommand{\hg}{{\hat{g}}}
\newcommand{\hw}{{\hat{w}}}
\newcommand{\hS}{{\hat{S}}}
\newcommand{\hV}{{\hat{V}}}
\newcommand{\hG}{{\hat{G}}}
\newcommand{\hmu}{{\hat{\mu}}}
\newcommand {\y}{\V{y}}
\newcommand {\V}[1]{\mbox{\boldmath$#1$}}
\newcommand{\iExp}{{\mathrm{iExp\,}}}

\newcommand{\htheta}{{\hat{\theta}}}



\newcommand{\htu}{{\hat{\tilde{u}}}}


\newcommand{\hTR}{{\widehat{TR}}}
\newcommand{\tsigma}{{\tilde{\sigma}}}
\newcommand{\tphi}{{\tilde{\phi}}}
\newcommand{\tpsi}{{\tilde{\psi}}}
\newcommand{\tzeta}{{\tilde{\zeta}}}
\newcommand{\tdelta}{{\tilde{\delta}}}
\newcommand{\tgamma}{{\tilde{\gamma}}}
\newcommand{\tGamma}{{\tilde{\Gamma}}}
\newcommand{\tlog}{{\widetilde{\log}}}


\newcommand{\txi}{{\tilde{\xi}}}
\newcommand{\tomega}{{\tilde{\omega}}}
\newcommand{\tH}{{\tilde{H}}}
\newcommand{\tI}{{\tilde{I}}}

\newcommand{\tX}{{\tilde{X}}}
\newcommand{\tV}{{\tilde{V}}}
\newcommand{\tz}{{\tilde{z}}}
\newcommand{\ty}{{\tilde{y}}}
\newcommand{\tx}{{\tilde{x}}}
\newcommand{\te}{{\tilde{e}}}
\newcommand{\tf}{{\tilde{f}}}
\newcommand{\tg}{{\tilde{g}}}
\newcommand{\tu}{{\tilde{u}}}
\newcommand{\tm}{{\tilde{m}}}
\newcommand{\tn}{{\tilde{n}}}
\newcommand{\tilt}{{\tilde{t}}}
\newcommand{\tT}{{\tilde{T}}}
\newcommand{\tL}{{\tilde{L}}}
\newcommand{\tQ}{{\tilde{Q}}}
\newcommand{\tB}{{\tilde{B}}}
\newcommand{\tC}{{\tilde{C}}}
\newcommand{\tD}{{\tilde{D}}}
\newcommand{\tU}{{\tilde{U}}}
\newcommand{\utL}{{\underline{\tilde{L}}}}
\newcommand{\tF}{{\tilde{F}}}\newcommand{\tilh}{{\tilde{h}}}
\newcommand{\tk}{{\tilde{k}}}
\newcommand{\tv}{{\tilde{v}}}
\newcommand{\tw}{{\tilde{w}}}
\newcommand{\bx}{\mathbf x}
\newcommand{\bz}{\mathbf z}
\newcommand{\bu}{\mathbf u}
\newcommand{\bv}{\mathbf v}
\newcommand{\bt}{\mathbf t}
\newcommand{\bi}{\mathbf i}
\newcommand{\bj}{\mathbf j}
\newcommand{\bL}{\mathbf L}
\newcommand{\bN}{\mathbf N}
\newcommand{\bM}{\mathbf M}
\newcommand{\bB}{\mathbf B}
\newcommand{\bA}{\mathbf A}
\newcommand{\tbz}{{\tilde{\mathbf z}}}
\newcommand{\hbx}{\hat{\mathbf x}}
\newcommand{\tcO}{{\tilde{\mathcal{O}}}}
\newcommand{\tcC}{{\tilde{\mathcal{C}}}}
\newcommand{\ocC}{{\overline{\mathcal{C}}}}
\newcommand{\tcR}{{\tilde{\mathcal{R}}}}
\newcommand{\tcA}{{\tilde{\mathcal{A}}}}








\newcommand{\uL}{\underline L}
\newcommand{\uM}{\underline M}
\newcommand{\uE}{\underline E}

\newcommand{\chA}{\check A}
\newcommand{\chE}{\check E}
\newcommand{\chL}{\check L}
\newcommand{\chV}{\check V}
\newcommand{\chv}{\check v}
\newcommand{\chw}{\check w}
\newcommand{\chW}{\check W}
\newcommand{\chM}{\check M}
\newcommand{\chQ}{\check Q}
\newcommand{\chsigma}{\check\sigma}



\newcommand{\uchL}{\underline{\check L}}






\newcommand{\BA}{\mathbb{A}}  \newcommand{\BB}{\mathbb{B}}
\newcommand{\CC}{\mathbb{C}}  \newcommand{\EE}{\mathbb{E}}
\newcommand{\FF}{\mathbb{F}}  \newcommand{\HH}{\mathbb{H}}
\newcommand{\JJ}{\mathbb{J}}  \newcommand{\LL}{\mathbb{L}}
\newcommand{\NN}{\mathbb{N}}  \newcommand{\PP}{\mathbb{P}}
\newcommand{\QQ}{\mathbb{Q}}  \newcommand{\RR}{\mathbb{R}}
\newcommand{\TT}{\mathbb{T}}  \newcommand{\VV}{\mathbb{V}}
\newcommand{\XX}{\mathbb{X}}  \newcommand{\WW}{\mathbb{W}}
\newcommand{\ZZ}{\mathbb{Z}}

\newcommand{\FM}{\mathfrak{M}}
\newcommand{\fm}{\mathfrak{m}}


\newcommand{\isom}{\cong}
\newcommand{\Ext}{\operatorname{Ext}}
\newcommand{\Grass}{\operatorname{Grass}}
\newcommand{\coker}{\operatorname{coker}}
\newcommand{\Hilb}{\operatorname{Hilb}}
\newcommand{\Hom}{\operatorname{Hom}}
\newcommand{\Quot}{\operatorname{Quot}}
\newcommand{\Pic}{\operatorname{Pic}}
\newcommand{\NS}{\operatorname{NS}}
\newcommand{\Sym}{\operatorname{Sym}}
\newcommand{\id}{\operatorname{I}}
\newcommand{\im}{\operatorname{im}}
\newcommand{\surj}{\twoheadrightarrow}
\newcommand{\inj}{\hookrightarrow}
\newcommand{\gr}{\operatorname{gr}}
\newcommand{\rk}{\operatorname{rk}}
\newcommand{\reg}{\operatorname{reg}}
\newcommand{\wt}{\widetilde}
\newcommand{\del}{{\partial}}
\newcommand{\delb}{{\overline\partial}}

\newcommand{\oX}{{\overline X}}
\newcommand{\oD}{{\overline D}}
\newcommand{\ox}{{\overline x}}
\newcommand{\ow}{{\overline w}}
\newcommand{\oz}{{\overline z}}
\newcommand{\oh}{{\overline{h}}}
\newcommand{\oalpha}{{\overline \alpha}}





\newcommand{\Res}{\operatorname{Res}}
\newcommand{\ch}{\operatorname{ch}}
\newcommand{\tr}{\operatorname{tr}}
\newcommand{\pardeg}{\operatorname{par-deg}}
\newcommand{\ad}{{ad\,}}
\newcommand{\diag}{\operatorname{diag}}
\newcommand{\codim}{\operatorname{codim}}

\hyphenation{pa-ra-bo-lic}
\newcommand{\bbQ}{\mathbb{Q}}
\newcommand{\bbR}{\mathbb{R}}
\newcommand{\bbP}{\mathbb{P}}
\newcommand{\bbC}{\mathbb{C}}
\newcommand{\bbT}{\mathbb{T}}
\newcommand{\bbU}{\mathbb{U}}
\newcommand{\bbZ}{\mathbb{Z}}
\newcommand{\bbN}{\mathbb{N}}
\newcommand{\bbF}{\mathbb{F}}






\newtheorem{proposition}{Proposition}[section]
\newtheorem{theorem}[proposition]{Theorem}
\newtheorem{lemma}[proposition]{Lemma}
\newtheorem{conjecture}[proposition]{Conjecture}
\newtheorem{corollary}[proposition]{Corollary}


\theoremstyle{definition}
\newtheorem{definition}[proposition]{Definition}
\newtheorem{remark}[proposition]{Remark}
\newtheorem{notation}[proposition]{Notation}
\newtheorem{example}[proposition]{Example}
\newtheorem{ex}{Exercise}[section]
 

%%%%%%%%%%%%%%%%%%%%%%%%%%%%%%%%%%%%%%%%%%%%%%%%%%%%%%%%

\begin{document}

\title{CMI UG Calculus 2020 Assignment 4 (draft)}
\date{\today}
\maketitle

\begin{enumerate}[wide, labelwidth=!, labelindent=0pt]

\item (Comparison of two ways of defining absolutely integrable functions on $\bbR$)
As preparation, consider the cover given by the open intervals $(n-1,n+1), \ n \in \bbZ$. Let $\psi$ be the function defined by:
\[
\psi(x)=
\begin{cases}
0 \ &if \ |x| \ge \frac{3}{4}\\
1 \ &if \ |x| \le \frac{1}{4}\\
2 \{\frac{3}{4}-t\} \ &if \ \frac{1}{4} \le x \le \frac{3}{4}\\
2\{t+\frac{3}{4}\} \ &if \ -\frac{3}{4} \le x \le -\frac{1}{4}\\
\end{cases}
\]
Note that $supp(\psi)=[-3/4,3/4]$.
Define functions $\psi_n, \ n \in \bbZ$ by:
\[
\psi_n(x)=\psi(x-n)
\]
Verify that the $\psi_n$ form a partition of unity, and in fact, $supp(\psi_n) \subset (n-1,n+1)$.

Let $f$ be a real-valued function on $\bbR$. Prove that the following are equivalent.
\begin{enumerate}
\item $f|_{[-m,m]}$ is integrable for each $m \in \bbN$ and the increasing sequence
\[
\int_{[-m,m]} |f(x)| dx
\]
tends to a finite limit.
\item $f|_{[m_1,m_2]}$ is integrable for each $m_1 < m_2 \in \bbZ$ and the set
\[
\int_{[m_1,m_2]} |f(x)| dx
\]
is bounded above.
\item $f|_{[n-1,n+1]}$ is  integrable for each $n \in \bbZ$ and the series
\[
\sum_n \int_{[n-1,n+1]} \psi_n (x) |f (x)| dx
\]
is convergent.
\end{enumerate}
If any one of the above conditions is met, we say that $f$ is absolutely integrable on $\bbR$. If this is the case, prove that the limits
\[
\lim_{m \to \infty} \int_{[-m,m]} f(x) dx
\]
and
\[
\lim_{m \to \infty} \sum_{n \le |m|} \int_{[n-1,n+1]} \psi_n(x) f(x) dx
\]
exist, and are equal.

Note that in the above expressions we could make replacements
\[
\begin{split}
\int_{[n-1,n+1]} \psi_n (x) |f (x)| dx &\Rightarrow \int_{\bbR} \psi_n (x) |f (x)| dx\\
\int_{[n-1,n+1]} \psi_n (x) f (x) dx &\Rightarrow \int_{\bbR} \psi_n (x) f(x) dx
\end{split}
\]
the right-hand sides being so defined. 

\item Now that we know how to define integrals of (absolutely integrable) functions on open subsets of $\bbR^n$, let us revisit the Gaussian integral. Consider the function $G_2(x,y)=\exp(-x^2-y^2)$ on $\bbR^2$. Most of the statements below are reformulations, or follow easily from, things that have already been proved. 
\begin{enumerate}
\item $G_2$ is absolutely integrable on $\bbR^2$.
\item The integral of $G_2$ over $\bbR^2$ is the same as its restriction to the complement of $\{(x,0)|x<0\}$.
\item The map $(r,\theta) \mapsto (r\cos \theta, r \sin \theta)$ is a $C^1$ (in fact $C^\infty$) diffeomorphism
\[
\{(r,\theta)|r>0,0 < \theta < 2\pi\} \longrightarrow \bbR^2 \setminus \{(x,0)|x<0\}
\]
\end{enumerate}

\item Do problems 7,8,9 of Rudin's book, Chapter 6.

\item Do problem 3-37 of Spivak.

\item Any course on integration should deal with lengths, areas and volumes. We may not have time to deal with areas and volumes. What follows is the content of \S 6.26 of Rudin.

A \emph{curve} in $\bbR^n$ is a continuous map $\gamma:[a,b] \to \bbR^n$. Given a partition $P$, given by a sequence $t_0,\dots,t_k$ such that $a=t_0 < t_1 <\dots <t_k=b$, we define
\[
\Lambda(P,\gamma)=\sum_{i=1}^k |\gamma(t_i)  -\gamma(t_{i-1})|
\]
where $|\gamma(t_i)  -\gamma(t_{i-1})|$ is the Euclidean distance in $\bbR^n$ between $\gamma(t_i)$ and  $\gamma(t_{i-1})$. We say that $\gamma$ is \emph{rectifiable} if
\[
length(\gamma) \equiv \underset{P}\sup\ \Lambda(P,\gamma) < \infty
\]

Prove that if $\gamma$ is a $C^1$ map, then it is rectifiable and
\[
length(\gamma) = \int_a^b |\dot{\gamma}(t)|dt
\]

For example, if $n=2$, and $\gamma(t)=(x(t),y(t))$
\[
length(\gamma) = \int_a^b \sqrt{|\dot{x}(t)|^2+|\dot{y}(t)|^2}dt
\]

\item  Problem 19 of Rudin, Chapter 6. This shows that the length of a path is independent of parametrisation.

\item Problem 19 of Rudin, Chapter 6,  assuming the parametrisations are $C^1$, and using the  change of variables formula for integration.

\end{enumerate}
 

 
 \end{document}

 