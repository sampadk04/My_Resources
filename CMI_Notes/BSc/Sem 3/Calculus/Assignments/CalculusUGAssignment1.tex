 \documentclass[11pt]{amsart}


\usepackage[all]{xy}
\usepackage{graphics}
\usepackage{enumitem}
\usepackage{epsfig}
\usepackage{amsmath}
\usepackage{amscd}
\usepackage{tikz-cd}
\usepackage{verbatim}
\usepackage{pdfpages}
%\usepackage{showkeys}
\usepackage{amsfonts,latexsym,amssymb}
\usepackage{parskip}
\usepackage{MnSymbol}
\usepackage{hyperref}

\usepackage{mdwlist}


%%%%%%%%%%%%%%%%%%%%%%%%%%%%%%%%%%%%%%%%%%%%%%%%%%%%%%%%


\newcommand{\cA}{{\mathcal{A}}}   \newcommand{\cB}{{\mathcal{B}}}
\newcommand{\cC}{{\mathcal{C}}}   \newcommand{\cD}{{\mathcal{D}}}
\newcommand{\cE}{{\mathcal{E}}}   \newcommand{\cF}{{\mathcal{F}}}
\newcommand{\cG}{{\mathcal{G}}}   \newcommand{\cH}{{\mathcal{H}}}
\newcommand{\cI}{{\mathcal{I}}}   \newcommand{\cJ}{{\mathcal{J}}}
\newcommand{\cK}{{\mathcal{K}}}   \newcommand{\cL}{{\mathcal{L}}}
\newcommand{\cM}{{\mathcal{M}}}   \newcommand{\cN}{{\mathcal{N}}}
\newcommand{\cO}{{\mathcal{O}}}   \newcommand{\cP}{{\mathcal{P}}}
\newcommand{\cQ}{{\mathcal{Q}}}   \newcommand{\cR}{{\mathcal{R}}}
\newcommand{\cS}{{\mathcal{S}}}   \newcommand{\cT}{{\mathcal{T}}}
\newcommand{\cU}{{\mathcal{U}}}   \newcommand{\cV}{{\mathcal{V}}}
\newcommand{\cW}{{\mathcal{W}}}   \newcommand{\cX}{{\mathcal{X}}}
\newcommand{\cY}{{\mathcal{Y}}}   \newcommand{\cZ}{{\mathcal{Z}}}

\newcommand{\hcP}{\hat{\mathcal{P}}}
\newcommand{\hcQ}{\hat{\mathcal{Q}}}
\newcommand{\hcR}{\hat{\mathcal{R}}}
\newcommand{\hcL}{\hat{\mathcal{L}}}
\newcommand{\hcM}{\hat{\mathcal{M}}} \newcommand{\hphi}{\hat{\phi}}
\newcommand{\bbk}{\mathbb{k}}   \newcommand{\bfv}{\mathbf{v}}
\newcommand{\bfnu}{\mathbf{nu}}  \newcommand{\hXX}{\hat{\mathbb{X}}}
\newcommand{\bJ}{\mathbf{J}}

\newcommand{\hD}{{\hat{D}}}   \newcommand{\hE}{{\hat{E}}}
\newcommand{\hF}{{\hat{F}}}   \newcommand{\hH}{{\hat{H}}}
\newcommand{\hY}{{\hat{Y}}}   \newcommand{\hP}{{\hat{P}}}
\newcommand{\hT}{{\hat{T}}}   \newcommand{\hQ}{{\hat{Q}}}
\newcommand{\hq}{{\hat{q}}}
\newcommand{\hr}{{\hat{r}}}
\newcommand{\hu}{{\hat{u}}}
\newcommand{\hv}{{\hat{v}}}
\newcommand{\hf}{{\hat{f}}}
\newcommand{\hg}{{\hat{g}}}
\newcommand{\hw}{{\hat{w}}}
\newcommand{\hS}{{\hat{S}}}
\newcommand{\hV}{{\hat{V}}}
\newcommand{\hG}{{\hat{G}}}
\newcommand{\hmu}{{\hat{\mu}}}
\newcommand {\y}{\V{y}}
\newcommand {\V}[1]{\mbox{\boldmath$#1$}}
\newcommand{\iExp}{{\mathrm{iExp\,}}}

\newcommand{\htheta}{{\hat{\theta}}}



\newcommand{\htu}{{\hat{\tilde{u}}}}


\newcommand{\hTR}{{\widehat{TR}}}
\newcommand{\tsigma}{{\tilde{\sigma}}}
\newcommand{\tphi}{{\tilde{\phi}}}
\newcommand{\tpsi}{{\tilde{\psi}}}
\newcommand{\tzeta}{{\tilde{\zeta}}}
\newcommand{\tdelta}{{\tilde{\delta}}}
\newcommand{\tgamma}{{\tilde{\gamma}}}
\newcommand{\tGamma}{{\tilde{\Gamma}}}
\newcommand{\tlog}{{\widetilde{\log}}}


\newcommand{\txi}{{\tilde{\xi}}}
\newcommand{\tomega}{{\tilde{\omega}}}
\newcommand{\tH}{{\tilde{H}}}
\newcommand{\tI}{{\tilde{I}}}

\newcommand{\tX}{{\tilde{X}}}
\newcommand{\tV}{{\tilde{V}}}
\newcommand{\tz}{{\tilde{z}}}
\newcommand{\ty}{{\tilde{y}}}
\newcommand{\tx}{{\tilde{x}}}
\newcommand{\te}{{\tilde{e}}}
\newcommand{\tf}{{\tilde{f}}}
\newcommand{\tg}{{\tilde{g}}}
\newcommand{\tu}{{\tilde{u}}}
\newcommand{\tm}{{\tilde{m}}}
\newcommand{\tn}{{\tilde{n}}}
\newcommand{\tilt}{{\tilde{t}}}
\newcommand{\tT}{{\tilde{T}}}
\newcommand{\tL}{{\tilde{L}}}
\newcommand{\tQ}{{\tilde{Q}}}
\newcommand{\tB}{{\tilde{B}}}
\newcommand{\tC}{{\tilde{C}}}
\newcommand{\tD}{{\tilde{D}}}
\newcommand{\tU}{{\tilde{U}}}
\newcommand{\utL}{{\underline{\tilde{L}}}}
\newcommand{\tF}{{\tilde{F}}}\newcommand{\tilh}{{\tilde{h}}}
\newcommand{\tk}{{\tilde{k}}}
\newcommand{\tv}{{\tilde{v}}}
\newcommand{\tw}{{\tilde{w}}}
\newcommand{\bx}{\mathbf x}
\newcommand{\bz}{\mathbf z}
\newcommand{\bu}{\mathbf u}
\newcommand{\bv}{\mathbf v}
\newcommand{\bt}{\mathbf t}
\newcommand{\bi}{\mathbf i}
\newcommand{\bj}{\mathbf j}
\newcommand{\bL}{\mathbf L}
\newcommand{\bN}{\mathbf N}
\newcommand{\bM}{\mathbf M}
\newcommand{\bB}{\mathbf B}
\newcommand{\bA}{\mathbf A}
\newcommand{\tbz}{{\tilde{\mathbf z}}}
\newcommand{\hbx}{\hat{\mathbf x}}
\newcommand{\tcO}{{\tilde{\mathcal{O}}}}
\newcommand{\tcC}{{\tilde{\mathcal{C}}}}
\newcommand{\ocC}{{\overline{\mathcal{C}}}}
\newcommand{\tcR}{{\tilde{\mathcal{R}}}}
\newcommand{\tcA}{{\tilde{\mathcal{A}}}}








\newcommand{\uL}{\underline L}
\newcommand{\uM}{\underline M}
\newcommand{\uE}{\underline E}

\newcommand{\chA}{\check A}
\newcommand{\chE}{\check E}
\newcommand{\chL}{\check L}
\newcommand{\chV}{\check V}
\newcommand{\chv}{\check v}
\newcommand{\chw}{\check w}
\newcommand{\chW}{\check W}
\newcommand{\chM}{\check M}
\newcommand{\chQ}{\check Q}
\newcommand{\chsigma}{\check\sigma}



\newcommand{\uchL}{\underline{\check L}}






\newcommand{\BA}{\mathbb{A}}  \newcommand{\BB}{\mathbb{B}}
\newcommand{\CC}{\mathbb{C}}  \newcommand{\EE}{\mathbb{E}}
\newcommand{\FF}{\mathbb{F}}  \newcommand{\HH}{\mathbb{H}}
\newcommand{\JJ}{\mathbb{J}}  \newcommand{\LL}{\mathbb{L}}
\newcommand{\NN}{\mathbb{N}}  \newcommand{\PP}{\mathbb{P}}
\newcommand{\QQ}{\mathbb{Q}}  \newcommand{\RR}{\mathbb{R}}
\newcommand{\TT}{\mathbb{T}}  \newcommand{\VV}{\mathbb{V}}
\newcommand{\XX}{\mathbb{X}}  \newcommand{\WW}{\mathbb{W}}
\newcommand{\ZZ}{\mathbb{Z}}

\newcommand{\FM}{\mathfrak{M}}
\newcommand{\fm}{\mathfrak{m}}


\newcommand{\isom}{\cong}
\newcommand{\Ext}{\operatorname{Ext}}
\newcommand{\Grass}{\operatorname{Grass}}
\newcommand{\coker}{\operatorname{coker}}
\newcommand{\Hilb}{\operatorname{Hilb}}
\newcommand{\Hom}{\operatorname{Hom}}
\newcommand{\Quot}{\operatorname{Quot}}
\newcommand{\Pic}{\operatorname{Pic}}
\newcommand{\NS}{\operatorname{NS}}
\newcommand{\Sym}{\operatorname{Sym}}
\newcommand{\id}{\operatorname{I}}
\newcommand{\im}{\operatorname{im}}
\newcommand{\surj}{\twoheadrightarrow}
\newcommand{\inj}{\hookrightarrow}
\newcommand{\gr}{\operatorname{gr}}
\newcommand{\rk}{\operatorname{rk}}
\newcommand{\reg}{\operatorname{reg}}
\newcommand{\wt}{\widetilde}
\newcommand{\del}{{\partial}}
\newcommand{\delb}{{\overline\partial}}

\newcommand{\oX}{{\overline X}}
\newcommand{\oD}{{\overline D}}
\newcommand{\ox}{{\overline x}}
\newcommand{\ow}{{\overline w}}
\newcommand{\oz}{{\overline z}}
\newcommand{\oh}{{\overline{h}}}
\newcommand{\oalpha}{{\overline \alpha}}





\newcommand{\Res}{\operatorname{Res}}
\newcommand{\ch}{\operatorname{ch}}
\newcommand{\tr}{\operatorname{tr}}
\newcommand{\pardeg}{\operatorname{par-deg}}
\newcommand{\ad}{{ad\,}}
\newcommand{\diag}{\operatorname{diag}}
\newcommand{\codim}{\operatorname{codim}}

\hyphenation{pa-ra-bo-lic}
\newcommand{\bbQ}{\mathbb{Q}}
\newcommand{\bbR}{\mathbb{R}}
\newcommand{\bbP}{\mathbb{P}}
\newcommand{\bbC}{\mathbb{C}}
\newcommand{\bbT}{\mathbb{T}}
\newcommand{\bbU}{\mathbb{U}}
\newcommand{\bbZ}{\mathbb{Z}}
\newcommand{\bbN}{\mathbb{N}}
\newcommand{\bbF}{\mathbb{F}}






\newtheorem{proposition}{Proposition}[section]
\newtheorem{theorem}[proposition]{Theorem}
\newtheorem{lemma}[proposition]{Lemma}
\newtheorem{conjecture}[proposition]{Conjecture}
\newtheorem{corollary}[proposition]{Corollary}


\theoremstyle{definition}
\newtheorem{definition}[proposition]{Definition}
\newtheorem{remark}[proposition]{Remark}
\newtheorem{notation}[proposition]{Notation}
\newtheorem{example}[proposition]{Example}
\newtheorem{ex}{Exercise}[section]
 

%%%%%%%%%%%%%%%%%%%%%%%%%%%%%%%%%%%%%%%%%%%%%%%%%%%%%%%%

\begin{document}

\title{CMI UG Calculus 2020 Assignment\\
 {\tiny One-dimensional Riemann integrals, \underline{due 25 August}}}
\date{\today}
\maketitle




\begin{enumerate}[wide, labelwidth=!, labelindent=0pt]
\item Let $S \subset \bbR$ be bounded above. Prove: $a = \sup S$ iff 
\begin{enumerate}[label=(\alph*)]
\item $a \ge x$ for all $x \in S$, and
\item there exists a sequence $x_n$ of elements in $S$ such that $\lim_n x_n=a$. 
\end{enumerate}

\item  Let $a<b$ be real numbers. Determine $\sup S$ and $\inf S$ in the following cases (justify the answer in the \emph{second} case): $S=[a,b],\  S=[a,b),\ S=(a,b],\ S=(a,b)$. 

\item Determine $\sup S$ and $\inf S$ in the following cases: 
\begin{enumerate}[label=(\alph*)]
\item $S=\{x \in \bbQ|x^2 \le 2\}$,
\item $S=\{x \in \bbQ|x^2 < 2\}$.
\item $S=\{x \in \bbQ|x > 0, x^2 \le 2\}$,
\item $S=\{x \in \bbR|x > 0, x^2 \le 2\}$
\end{enumerate}
 

\item Which of the following functions on $(-1,1)$ are uniformly continuous? 

\begin{enumerate}[label=(\alph*)]
\item \[
f(x)=
\begin{cases}
1\ \ if \ \ \textup{$x \ge 0$}\\
-1 \ \ if \ \ \textup{$x < 0$}
\end{cases}
\]
\item $f(x)=x$
\item $f(x)=\tan \frac{\pi x}{2}$
\end{enumerate}



\item Is the function $\log  x$ uniformly continuous on $[1,\infty)$? 

\item  Let $f$ a continuously differentiable\footnote{Recall that this means: $f$ is differentiable and $f'$ is continuous.} function defined on an \emph{open} interval.  What is a natural condition on $f'$ that will guarantee that $f$ is uniformly continuous? Justify your answer. 

\item Prove the following statements. (This is essentially Theorem 6.12 of Rudin.)

\begin{enumerate}[label=(\alph*)]
\item If $f,g$ are Riemann integrable on $[a,b]$, then so are $f+g$, and $fg$, and
\begin{equation*}
\begin{split}
\int_a^b (f+g)(x) dx&= \int_a^b f(x) dx +\int_a^b g(x) dx\\
\int_a^b cf(x) dx &= c\int_a^b f(x) dx\ \textup{if $c$ is a constant.}
\end{split}
\end{equation*}

\item If $f,g$ are Riemann integrable on $I = [a,b]$ and $f(x) \le g(x), x \in I$, then
\[
\int_a^b f(x) dx  \le \int_a^b g(x) dx
\]

\item If $f$ is Riemann integrable on $[a,b]$ and $a < c <b$, then (the restriction of)   $f$ is Riemann integrable on $[a,c]$ and $[c,b]$ and
\[
\int_a^b f(x) dx= \int_a^c f(x) dx+\int_c^b f(x) dx
\]

\item Let us agree that  \emph{a complex-valued function $f=f_r+if_i$ defined on $I=[a,b]$ will be called Riemann integrable} if its real and imaginary parts $f_r,f_i$ are Riemann Integrable; in which case, we define
\[
\int_a^b f(x) dx  \equiv  \int_a^b f_r(x) dx + i \int_a^b f_i(x) dx \ .|
\]
Prove that $f$ is Riemann integrable iff $|f|$ is Riemann integrable, and that
\[
|\int_a^b f(x) dx| \le \int_a^b |f(x)| dx
\]

\end{enumerate} 

The following problems are (with minor changes) taken from Rudin. 

\item (Holder's inequality) Let $p,q$ be positive real numbers satisfying
\[
\frac{1}{p} + \frac{1}{q} =1 
\]
(These  are said to be \emph{conjugate exponents} to each other. Note that $p=2,q=2$ are conjugate. Note also the ``limiting cases'' $p=1,q=\infty$, $p=\infty, q=1$.)
\begin{enumerate}[label=(\alph*)]
\item If $u \ge 0, \ v \ge 0$, prove that
\[
uv \le \frac{u^p}{p}+\frac{v^q}{q}
\]
(Hint: Reduce to proving the case when $u=1$ and $0 \le v \le 1$. When $v=0$ or $v=1$, the inequality is clear; now use convexity. )
\item  If $f,g$ are Riemann integrable non-negative functions, then
\[
\int_a^b fg dx  \le \{\int_a^b f^p dx\}^{\frac{1}{p}} \{\int_a^b g^q dx\}^{\frac{1}{q}}
\]
(Hint: Reduce to the case when both factors on the right are equal to one. Then use (a).)
\item If $f,g$ are complex-valued and Riemann integrable,
\[
|\int_a^b fg dx|  \le \{\int_a^b |f|^p dx\}^{\frac{1}{p}} \{\int_a^b |g|^q dx\}^{\frac{1}{q}}
\]

\end{enumerate} 

\item For Riemann integrable (possibly complex-valued) $u$, define
\[
||u||_2 \equiv \{\int_a^b |u(x)|^2 dx\}^\frac{1}{2}
\]
Prove that if $f,g,h$ are Riemann integrable,
\[
||f-h||_2 \le ||f-g||_2+||g-h||_2
\]



\end{enumerate}



\end{document}

 