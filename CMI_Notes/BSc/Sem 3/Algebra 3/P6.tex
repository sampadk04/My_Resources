\documentclass[12pt,a4paper]{article}
\usepackage[utf8]{inputenc}
\usepackage{graphicx}
\usepackage{graphics}
\graphicspath{./images/}
\usepackage{url}
\usepackage{amsmath,amsfonts,amsthm,amssymb,color,amsbsy}
\usepackage[pagebackref=true,colorlinks]{hyperref}
\usepackage{tikz,pgf}
\usepackage{tikz-cd}
\usepackage{mathrsfs}
\usepackage{xlop}
\usepackage{soul}
\usepackage{xcolor}
\graphicspath{ {./Images/} }
\newtheorem{theorem}{Theorem}[section]
\newtheorem{corollary}[theorem]{Corollary}
\newtheorem{lemma}[theorem]{Lemma}

\addtolength{\oddsidemargin}{-.75in}
\addtolength{\evensidemargin}{-.75in}
\addtolength{\textwidth}{1.75in}

\addtolength{\topmargin}{-.875in}
\addtolength{\textheight}{1.75in}
\hypersetup{
	colorlinks=true,
	linkcolor=blue,
	citecolor=magenta
}


\definecolor{xdxdff}{rgb}{0.49019607843137253,0.49019607843137253,1.}
\definecolor{zzttqq}{rgb}{0.6,0.2,0.}
\definecolor{qqqqff}{rgb}{0.,0.,1.}

\newtheorem{proposition}[theorem]{Proposition}
\newtheorem{conjecture}[theorem]{Conjecture}
\newtheorem{claim}[theorem]{Claim}
\newtheorem{fact}[theorem]{Fact}
\newtheorem{assumption}[theorem]{Assumption}
\newtheorem{warning}[theorem]{Warning}

\theoremstyle{definition}
\newtheorem{definition}[theorem]{Definition}
\newtheorem{example}[theorem]{Example}
\newtheorem{remark}[theorem]{Remark}
\newtheorem{exercise}[theorem]{Exercise}
\newtheorem{recall}[theorem]{Recall}
\newtheorem{observation}[theorem]{Observation}

\newcommand\numberthis{\addtocounter{equation}{1}\tag{\theequation}}
\newcommand{\F}{\mathbb{F}}
\newcommand{\Z}{\mathbb{Z}}
\newcommand{\N}{\mathbb{N}}
\newcommand{\Q}{\mathbb{Q}}
\newcommand{\R}{\mathbb{R}}
\newcommand{\C}{\mathbb{C}}
\newcommand{\K}{\mathbb{K}}
\newcommand{\p}{\mathbb{P}}
\newcommand{\e}{\epsilon}
\newcommand{\A}{\mathcal{A}}
\newcommand{\V}{\mathbf{v}}
\newcommand{\X}{\mathcal{X}}
\newcommand{\n}{\mathcal{N}}
\DeclareMathOperator{\im}{im}
\DeclareMathOperator{\sgn}{sgn}
\DeclareMathOperator{\tr}{tr}
\DeclareMathOperator{\adj}{adj}
\DeclareMathOperator{\real}{Re}
\DeclareMathOperator{\imag}{Im}
\newcommand{\bigO}{\mathcal{O}}
\newcommand{\ds}{\displaystyle}
\newcommand{\bs}{\boldsymbol}
\newcommand{\cl}{\overline}
\newcommand{\inr}[1]{\left\langle #1 \right\rangle}
\newcommand{\nrm}[1]{\left\| #1 \right\|}
\newcommand{\abs}[1]{\left| #1 \right|}
\newcommand{\set}[1]{\left{ #1 \right}}
\newcommand{\mat}[1]{\begin{bmatrix}#1\end{bmatrix}}

\title{Problem 6}
\author{Sampad Kumar Kar -- BMC201944}


\begin{document}

\maketitle

\begin{flushleft}

(i) Simplifying $\frac{\mathbb{F}_2[x]}{(x^{2^6}-x)}$.

\medskip

Firstly, we know that- $$x^{p^n} - x = \prod_{d|n} (\text{product of monic irreducible polynomials of degree 'd' in } \mathbb{F}_p[x])$$

Also, if $d|n$, then $x^{p^d}-x|x^{p^n}-x$.

\medskip

Using these results and Chinese Remainder Theorem, we simplify.

In this problem $p = 2$ and $n = 6$.

So, we will find the no. of monic irreducibles of degree 1, 2, 3 and 6 in $\mathbb{F}_2[x]$.

\begin{itemize}
	\item degree 1

	The only monic factors are $x$ and $x-1$, as there are no other roots possible. So, we have 2.

	\item degree 2

	For this we consider $x^{2^2} - x$. All monic irreducibles of degree 2 divide this. But this is the product of all monic irreducibles of degree 1 and 2. Let the total no. of deg 2 irreducibles be $y$.

	So, $2y + 2 = 2^2 = 4 \implies y = 1$, meaning we have only 1.

	\item degree 3

	We consider $x^{2^3}-x$ for this. All monic irreducibles of degree 3 divide this. But this is the product of all monic irreducibles of degree 1 and 3. Let the total no. of deg 3 irreducibles be $y$.

	So, $3y + 2 = 2^3 = 8 \implies y = 2$, meaning we have 2.

	\item degree 6

	We consider $x^{2^6} -x$ for this. All monic irreducibles of degree 6 divide this. But this is the product of all monic irreducibles of degree 1, 2, 3, and 6. Let the total no. of deg 6 irreducibles be $y$.

	So, $6y + 3.2 + 2.1 + 1.2 = 2^6 = 64 \implies y = 9$, meaning we have 9.
\end{itemize}

Now, $\frac{\mathbb{F}_2[x]}{(f(x))} \cong \mathbb{F}_{2^n}$, where $f(x)$ is an irreducible of degree $n$.

\medskip

Now, we use Chinese Remainder Theorem for all the irreducible factors of $x^{2^6} - x$ to obtain-

$$\frac{\mathbb{F}_2[x]}{(x^{2^6}-x)} \cong (\mathbb{F}_{2})^2 \times (\mathbb{F}_{2^2}) \times (\mathbb{F}_{2^3})^2 \times (\mathbb{F}_{2^6})^9$$

\end{flushleft}
\end{document}