
\documentclass{article}
\usepackage[utf8]{inputenc}
\usepackage{amsmath}
\usepackage{amssymb}
\usepackage{enumerate}
\usepackage[a4paper, total={6.5in, 9.5in}]{geometry}

\begin{document}
\parindent = 0pt
{\centerline {\bf Your name here}}

\medskip

{\bf Algebra III Homework 0} 
\hfill {\bf Due by 11.59 pm on Sunday Sep 13}

\bigskip

 \fbox{%
    \parbox{\dimexpr\linewidth-2\fboxsep}{%
This \TeX~file is a simple hack job by a novice. Feel free to incinerate everything except the text of the problem and sculpt your own beautiful creation from scratch. Or not. But the lower bound is that your work must be clearly identifiable and pleasant to read.
    }%
}  
\bigskip

{\bf 1.}  Let $a$ be a nonzero integer and let $b$ be an integer. Carry out the following step-by-step development. Useful notation: if there is an integer $c$ with $b = ac$, then we write $a|b$ and in words we say ``$a$ is factor of $b$" or ``$a$ divides $b$" or ``$b$ is divisible by $a$". 

\begin{enumerate}[(i)]

    \item {\bf Long division.} Carefully state the meaning of saying that we can do long division of $b$ by $a$ to get quotient $q$ and remainder $r$. This is crucial to all that follows.

 \fbox{%
    \parbox{\dimexpr\linewidth-2\fboxsep}{%
In general put your answer to each question/subpart directly below that part. You could put each answer in a box like this so that your work stands out easily.
    }%
}  
    \item {\bf Meaning of GCD.} Define what it means for an integer $d$ to be a $gcd$ of $a$ and $b$. Try to capture the notion of ``greatest" using only divisibility, not size. Defined this way, $gcd(a,b)$ is {\it essentially} unique. Explain how.
    
    \item {\bf Euclidean algorithm.} Use long division to prove existence of $gcd(a,b)$ and to calculate it. Note that at this stage you do NOT know anything about primes, much less about prime factorization. See step (\ref{UFD}) below.
    
    \item {\bf GCD as linear combination.} Why can $gcd(a,b)$ be written in the form $xa + yb$ and how can we find such integers $x$ and $y$? To what extent are $x$ and $y$ unique?
    
    \item {\bf Two ways to think of a prime number.} Suppose $p$ is an integer other than $0,\pm 1$. Show the equivalence: ``$p$ has no factors other than $\pm p$ and $\pm 1$" $\Leftrightarrow$  ``if $p|ab$ then $p|a$ or $p|b$". Under these conditions we call $p$ a prime number.
    
    \item {\bf Existence of prime factorization.} Why can each nonzero integer $n$ other than $\pm 1$ be written as a finite product of prime numbers?
    
    \item \label{UFD} {\bf Uniqueness of prime factorization.} In what way is an expression of $n$ as a product of primes unique? Formulate this carefully and prove it.
    
    \item {\bf Chinese remainder theorem.} Suppose $gcd(a,b)=1$. Prove that given any integer $r$ and any integer $s$, there exists an integer $n$ such that $a|n-r$ and $b|n-s$. How does one explicitly find such $n$? To what extent is it unique?
    
    \item {\bf Example.} Carry out parts (iii) and (iv) for $(a,b) = ($your roll number of form 201xxx, 2017). Then do part (viii) for $(r,s) = (20,19)$.

\end{enumerate}

\end{document}