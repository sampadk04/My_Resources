
\documentclass{article}
\usepackage[utf8]{inputenc}
\usepackage{amsmath}
\usepackage{amssymb}
\usepackage{enumerate}
\usepackage[a4paper, total={6.5in, 9.5in}]{geometry}

\begin{document}
\parindent = 0pt
{\centerline {\bf Your name here}}

\medskip

{\bf Algebra III Homework 1} 
\hfill {\bf Due by 11.59 pm on Wed Nov 4}

\bigskip

{\bf 3. Do NOT submit this.} Consider a set $S$ with binary operations $+$ and $\cdot$ satisfying all axioms of a ring {\it except} commutativity of addition and existence of 1. Multiplication is not assumed to be commutative. Show that any two elements of the subset $\{a \cdot b \, | \, a,b \in S\}$ automatically commute under addition. Deduce that for a ring with 1, commutativity of addition follows from the rest of the axioms. (This is a version of the Eckmann-Hilton argument, look it up if you want to know more.)

\medskip
   
{\bf 4.} {\bf Funny ring structure.}  Given a ring $R$, show that we get a new ring structure on the same {\it set} $R$ as follows: define a new addition $\oplus$ by $a \oplus b = a + b - 1$ and and a new multiplication $\odot$ by $a \odot b = a + b - ab$. You can prove this by going through the axioms, {\it but that is not the point of this exercise at all}. Instead prove the claim by simultaneously showing that the new ring is actually {\it isomorphic} to the original ring. (The basic observation is that if there is any {\it bijection} $f$ from a group/vector space/ring/whatever structure to some set $S$, then one can make $S$ into the same structure by using $f$ as a dictionary.)

\medskip

 \fbox{%
    \parbox{\dimexpr\linewidth-2\fboxsep}{%
In general put your answer to each question/subpart directly below that part. You could put each answer in a box like this so that your work stands out easily.
    }%
}  

\medskip

{\bf 5.} Artin Chapter 11: 3.3 c and e on kernel of maps from polynomial rings. Find as few (and as simple) generators as you can.

\medskip

{\bf 6.} Artin Chapter 11: 3.6 and 3.7 on ring automorphisms of $R[x,y]$ and of $\Bbb Z[x]$. Do these exercises cleanly by using substitution principle as much as you can. While it is true that an isomorphism is a bijective ring homomorphism, it may be better to think of it equivalently as a (ring) map $f$ such that there is an inverse (ring) map $g$ in the opposite direction, i.e., such that  $f \circ g$ and $g \circ f$ are the respective identity maps.

\medskip

Automorphisms of polynomial rings remain a mystery in general. Here is a classic (3-page!) paper: 
{\it Stably tame automorphisms, by Martha K. Smith, Journal of Pure and Applied Algebra 58 (1989) 209-212.} 
Let me know if you read it.

\medskip

{\bf 7.}  Artin Chapter 11: 3.9 on nilpotent/unipotent elements. You may appeal to the very standard Frobenius map in exercise 3.8, but prove it for yourself! (The definition of unipotent used here is non-standard. Usual ring theory definition is unipotent = 1 + nilpotent, but here use the given definition.)

\medskip

{\bf 8.}	Artin Chapter 11: 2.2 on units in $F[[t]]$ +  3.10 on ideals in $F[[t]]$. Which of the ideals you found are maximal? Which are prime?

\medskip

{\bf 9.}  For $c$ in a ring $D$ and nonzero $f(x)$ in $D[x]$, define the ``multiplicity of $c$ as a root of $f(x)$" to be the largest nonnegative integer $n$ such that $f(x) = (x-c)^n q(x)$ in $D[x]$. Observe that this is well-defined. 

\medskip

(i) In a {\it domain} $D$, show that $\prod_{c \text { root of } f(x)} \, (x - c )^{\text {multiplicity of } c \text { as a root of } f(x)}$ is a factor of  $f(x)$. This generalizes an exercise done in the lecture.

\medskip
	
(ii) Find a counterexample to (i) where is $D$ not a domain and $f$ is a {\it monic} polynomial with a root whose multiplicity equals your roll number.

\medskip

(iii)	For a finite field $F$ of cardinality $q$, show that $x^q - x  = \prod_{c \in F} \, (x - c)$. You do NOT need to use the fact that $F[x]$ has unique factorization property. (Hint: the set of nonzero elements in $F$ forms a group under multiplication. The order of each element of a finite group is a factor of the order of the group.)

\end{document}
