
\documentclass{article}
\usepackage[utf8]{inputenc}
\usepackage{amsmath}
\usepackage{amssymb}
\usepackage{enumerate}
\usepackage[a4paper, total={6.5in, 9.5in}]{geometry}

\begin{document}
\parindent = 0pt
{\centerline {\bf Your name here}}

\medskip

{\bf HW 2}: problems 11 to 16  below  
\hfill {\bf Due by 11.59 pm on Wed Nov  11}

\bigskip

{\bf 10. Do not submit this.}  Artin Chapter 11:  4.4. Find as many arguments as you can, e.g. look up on stackexchange. Generally if you look for help, STOP as soon as you read/hear the first {\it word} containing an idea that did not occur to you. See if you can take it from there on your own. Then try to analyze how the given problem could have suggested this idea to you. This is for future use and the whole point of the problem! Keep such ideas squirrelled away in your mind, to be deployed as and when applicable.

\medskip

{\bf 11.}  Artin Chapter 11: 4.3 b and e (Here take ``identify" to mean ``find cardinality and whether the ring is a field/integral domain" You can say more if you like.) 

\medskip

 \fbox{%
    \parbox{\dimexpr\linewidth-2\fboxsep}{%
In general put your answer to each question/subpart directly below that part. You could put each answer in a box like this so that your work stands out easily.
    }%
}  

\medskip

{\bf 12.}  Artin Chapter 11: 5.3

\medskip

{\bf 13.}  Artin Chapter 11: 5.4 a and b

\medskip

{\bf 14.}  Artin Chapter 11: 5.5 (Hint: consider maximal ideals.)

\medskip

{\bf 15.}  Artin Chapter 11: 8.2 b, c and d, also identify which of the given rings are fields.

\medskip

{\bf 16.}  Artin Chapter 11: Suppose you are {\it given} a finite field $E$. Show that $ |E| = p^n$ for a prime number $p$. (Hints: (i) First identify $p$ from $F$ using the first isomorphism theorem. Which other ring should you use? (ii) If a field $F$ is subfield of a ring $E$, then note that $E$ is in particular a vector space over $F$ with the given operations. We will use this repeatedly, especially when $E$ is a field as well. In particular we can consider dimension of $E$ over $F$, which we call the degree of $E$ over $F$, denoted $[E:F]$.)

\medskip

Keep a list of known facts for finite fields. So far you know (1)  that cardinality is always = $p^n$  for some prime $p$, (2) the Frobenius map from exercise 11.3.8, (3) that each element of such a field of cardinality $q$ (if such a field exists) is a root of $x^q - x$. Add to this list every time you see something new that can be applied and keep trying to show existence and uniqueness of a field of cardinality $p^n?$ Can you do it now?! %(Need more - existence: attaching roots! uniqueness: cyclicity of F cross)

\end{document}