
\documentclass{article}
\usepackage[utf8]{inputenc}
\usepackage{amsmath}
\usepackage{amssymb}
\usepackage{enumerate}
\usepackage[a4paper, total={6.5in, 9.5in}]{geometry}

\begin{document}
\parindent = 0pt
{\centerline {\bf Sampad Kumar Kar - BMC201944}}

\medskip

{\bf Algebra III Homework 0} 
\hfill {\bf Due by 11.59 pm on Sunday Sep 13}

\bigskip 
\bigskip

{\bf 1.}  Let $a$ be a nonzero integer and let $b$ be an integer. Carry out the following step-by-step development. Useful notation: if there is an integer $c$ with $b = ac$, then we write $a|b$ and in words we say ``$a$ is factor of $b$" or ``$a$ divides $b$" or ``$b$ is divisible by $a$". 
\bigskip
\begin{enumerate}[(i)]

    \item {\bf Long division.} Carefully state the meaning of saying that we can do long division of $b$ by $a$ to get quotient $q$ and remainder $r$. This is crucial to all that follows.

 \fbox{%
    \parbox{\dimexpr\linewidth-2\fboxsep}{%
\textbf{Solution (i):}

\textit{Euclid's Division Lemma} says that for $a , b \in \mathbb{Z}$ with $a \ne 0$, $\exists! q,r \in \mathbb{Z}$, such that $b = a.q + r$, where $0 \le r < |a|$.
    }%
}
\bigskip
    \item {\bf Meaning of GCD.} Define what it means for an integer $d$ to be a $gcd$ of $a$ and $b$. Try to capture the notion of ``greatest" using only divisibility, not size. Defined this way, $gcd(a,b)$ is {\it essentially} unique. Explain how.

 \fbox{%
    \parbox{\dimexpr\linewidth-2\fboxsep}{%
\textbf{Solution (ii):}

We say $d = gcd(a,b)$, if and only if $d>0$ and $d | a$, $d | c$ and for any $x$ such that $x | a$, $x | b$ implies $x | d$.

\smallskip

This definition makes $d$ \textit{essentially} unique because of the following reasoning-

Suppose $d_1$ and $d_2$ are $gcd$ of $a$ and $b$. This means $d_1|a$ and $d_1|b$. But as $d_2 = gcd(a,b)$, by our definition, $d_1|d_2$. Similarly, we can show that $d_2|d_1$, which implies $d_1 = d_2$.
    }%
}
\bigskip
    \item {\bf Euclidean algorithm.} Use long division to prove existence of $gcd(a,b)$ and to calculate it. Note that at this stage you do NOT know anything about primes, much less about prime factorization. See step (\ref{UFD}) below.

 \fbox{%
    \parbox{\dimexpr\linewidth-2\fboxsep}{%
\textbf{Solution (iii):}

Let $a \ne 0$. Using \textbf{(i)}, $\exists! q,r \in \mathbb{Z}$ such that $b = a.q + r$ and $0 \le r < |a|$.

\smallskip

We will first prove the \textit{Euclidean Algorithm}, which says $d = gcd(a,b)$ if and only if $d = gcd(r,a)$.

\smallskip

($\Longrightarrow$) Let $d = gcd(a,b)$. Now, $r = b - a.q$, which means $d | r$. Now, suppose $c | r$ and $c | a$. This means $c | (a.q + r) \implies c | b$, which means $c$ divides both $a$ and $b$, meaning $c | d$. Hence, $gcd(r,a) = d$ as well.

\smallskip

($\Longleftarrow$) Let $d = gcd(r,a)$. Now $b = a.q + r$, which means $d | b$. Now, suppose $c|a$ and $c|b$. Then $c|(b-a.q) \implies c|r$, which means $c|d$. Hence, $gcd(a,b) = d$ as well.

\smallskip

Now, we use this \textit{Algorithm}, to calculate the $gcd$, hence proving its existence as well.

\smallskip

Suppose we are to calculate $gcd(a,b)$. By the above \textit{Algorithm}, this should be equal to $gcd(r,a)$. Now, we recursively keep applying this \textit{Algorithm} to make the first input smaller and smaller, and after a finite no. of steps the first input should become $0$. Then $gcd(0,k) = k, k > 0$, and this is our required $gcd$ as per this \textit{Algorithm}. Hence, we are done.
    }%
}

\newpage
    \item {\bf GCD as linear combination.} Why can $gcd(a,b)$ be written in the form $xa + yb$ and how can we find such integers $x$ and $y$? To what extent are $x$ and $y$ unique?

 \fbox{%
    \parbox{\dimexpr\linewidth-2\fboxsep}{%
\textbf{Solution (iv):}

First, I will show that $gcd (a,b)$ can be written in the form of $ax + by$, where $x,y \in \mathbb{Z}$.

\smallskip

Consider the set $S = \{ax + by|x,y \in \mathbb{Z}, ax+by > 0\}$. Since, $S \ne \phi$, by \textit{Well Ordering Property}, $\exists$ a smallest element $d \in S$. Lets call this the $gcd$. Now, I will show that $d$ indeed satisfies the definition of $gcd$.

\smallskip

First, I will show that $d|a$ and $d|b$. Suppose not. W.L.O.G. assume $d \not| a$. Now, as $d \in S$, let $d = as + bt$ for some $s,t \in \mathbb{Z}$. Then by \textbf{(i)}, $\exists q,r \in \mathbb{Z}$, such that $a = q.d + r$, where $0 < r < d$ ($r$ can't be $0$ as $d \not| a$). This means-
$$a = d.q + r \implies r = a - d.q = a - (as + bt).q = a - asq - btq = a(1-sq) + b(-tq)$$
which means, $r \in S$. But this is a contradiction as $r < d$, and we assumed $d$ to be the smallest element of $S$. Hence, $d | a$. Analogously, we can show that $d | b$.

\smallskip

Now, I will show that if $c | a$ and $c | b$, then $c | d$. Clearly if such a $c$ exists, then $c | (ax + by)$ for all $x,y \in \mathbb{Z}$. This means $c | (as + bt) \implies c | d$. Hence, $d = gcd(a,b)$ and it can be represented as a linear combination of $a$ and $b$.

Now, we can find the values of $s$ and $t$, by reverse engineering the \textit{Euclid's Algorithm} to find $gcd$. 

\smallskip

For example- First, I find $gcd(201944,2017)$ using \textit{Euclidean Algorithm}, as portrayed in \textbf{(iii)}.

$$201944 = 100.2017 + 244$$
$$2017 = 8.244 + 65$$
$$244 = 3.65 + 49$$
$$65 = 1.49 + 16$$
$$49 = 3.16 + 1$$
$$16 = 16.1 + 0$$

Hence, $gcd(201944,2017) = 1$.

\bigskip

Now, I use these above equations to reverse engineer a $s$ and $t$ in $\mathbb{Z}$, such that $201944s + 2017t = 1$.

$$1 = 1.49 - 3.16$$
$$= 1.49 - 3.(65-1.49) = 4.49 - 3.65$$
$$= 4.(244-3.65) - 3.65 = 4.244 - 15.65$$
$$= 4.244 - 15.(2017 - 8.244) = 124.244 - 15.2017$$
$$= 124.(201944-100.2017) - 15.2017 = 124.201944 - 12415.2017$$

Hence $gcd(201944,2017) = 1 = 201944(124) + 2017(-12415)$. Here, $s = 124$ and $t = -12415$. So, by using this method, we can explicitly calculate $s$ and $t$ in $\mathbb{Z}$ such that $gcd(a,b) = as + bt$.

\smallskip

Now, I discuss the uniqueness of $s$ and $t$. Let $s'$ and $t'$ be such that $as' + bt' = d$.

\smallskip

Now, $gcd(a,b) = d$. Let $\frac{a}{d} = a'$ and $\frac{b}{d} = b'$, which means $gcd(a',b') = 1$.

Also, we know that, if $gcd(a,b) = 1$ and $a|bc \implies a|c$.
$$as' + bt' = d = as + bt \implies a's' + b't' = a's + b't \implies a'(s' - s) = b'(t - t')$$
which means $a'|b'(t-t') \implies a'|(t-t')$ (as $gcd(a',b') = 1$).

\smallskip

Let $t-t' = a'k \implies t' = t-a'k$. This also means $a'(s'-s) = b'a'k \implies s'-s = b'k \implies s' = s + b'k$. So, $$s'= s + k.\frac{b}{d} \text{ and } t' = t - k.\frac{a}{d}$$ for $k \in \mathbb{Z}$.
    }%
}
\newpage
    \item {\bf Two ways to think of a prime number.} Suppose $p$ is an integer other than $0,\pm 1$. Show the equivalence: ``$p$ has no factors other than $\pm p$ and $\pm 1$" $\Leftrightarrow$  ``if $p|ab$ then $p|a$ or $p|b$". Under these conditions we call $p$ a prime number.

 \fbox{%
    \parbox{\dimexpr\linewidth-2\fboxsep}{%
\textbf{Solution (v):}

($\Longrightarrow$) Suppose $p$ has no factors other than $\pm p$ and $\pm 1$.

\smallskip

Assume on the contrary that $p \not| a$ and $p \not| b$. Let $d = gcd(p,a)$. Since $d|p$, the only choices for $d$ are $1$ and $|p|$. Now $d \ne |p|$ as $p \not| a$. So, $d = 1$. But $p|ab$ and $gcd(p,a) = 1$, which means $p|b$, which is a contradiction. So, if $p|ab$ either $p|a$ or $p|b$.

\smallskip

($\Longleftarrow$) Suppose $p|ab \implies p|a \text{ or } p|b$. Clearly, $\pm p$ and $\pm 1$ are factors of $p$. I will, show that no other factor other than these, can exist for $p$.

\smallskip

Assume on the contrary that $\exists c \in \mathbb{Z}$, such that $c \ne \pm 1,\pm p$ and $c|p$. Let $d = \frac{p}{c} \implies p = c.d$. Now, this means $|c|,|d| < |p|$.

\smallskip

Consider $a = c$ and $b = d$. Clearly $ab = cd = p$ and $p | ab \implies p|a \text{ or } p|b$ which means $p|c \text{ or } p|d$. Now since, $p \ne 0,\pm 1$, this is a contradiction, as $|c|,|d| < |p|$. Hence, $c \not| p$. So, $\pm p \text(and) \pm 1$ are the only factors of $p$.
    }%
}
\bigskip
    \item {\bf Existence of prime factorization.} Why can each nonzero integer $n$ other than $\pm 1$ be written as a finite product of prime numbers?

 \fbox{%
    \parbox{\dimexpr\linewidth-2\fboxsep}{%
\textbf{Solution (vi):}

W.L.O.G. we can prove this for positive integers. To show existence of prime factorization, we use strong induction. Now, $2$ is already a prime itself, so we are done with the base case. Now, assume that we can express all integers $1 < x < n$ as product of primes. Now, we will show that $n$ can also be expressed as product of primes. For this we consider $2$ cases.

\smallskip

(i) If $n$ is prime: Then we are already done.

\smallskip

(ii) If $n$ is not prime: Then by \textbf{(v)}, $\exists a \in \mathbb{Z}$ such that $a|n$ and $a \ne \pm 1, \pm n$. Let $b = \frac{n}{a} \implies n = ab$. Then $a,b < n$. Now, by induction hypothesis, as $0 < a,b < n$, we can express $a$ and $b$ as product of one or more primes, and hence $n$ as product of one or more primes. Hence, we are done.
    }%
}
\bigskip
    \item \label{UFD} {\bf Uniqueness of prime factorization.} In what way is an expression of $n$ as a product of primes unique? Formulate this carefully and prove it.

 \fbox{%
    \parbox{\dimexpr\linewidth-2\fboxsep}{%
\textbf{Solution (vii):}

Suppose $n$ can be expressed as multiple prime product expressions in ordered fashion, say $n = p_1.p_2...p_k$ and $n = q_1.q_2...q_l$, where $p_i$ and $q_i$ are primes. Since, $p_1.p_2...p_k = q_1.q_2...q_l$, for any $1 \le i \le k$, $p_i|q_1.q_2...q_l$. So, by \textbf{(v)}, $\exists q_j, 1 \le j \le l$, such that $p_i|q_j$. Since, $p_i$ and $q_j$ are primes, it follows that $p_i = q_j$. Continuing this process, we can cancel out all the common prime factors.

\smallskip

If $k \ne l$, we will either have $p_{i_1}.p_{i_2}...p_{i_s} = 1$ or $1 = q_{j_1}.q_{j_2}...q_{j_t}$. Either ways, we arrive at a contradiction. This means $k = l$. Since, every time we cancel out primes, we take out some $p_i$ equals to some $q_j$, this means both these expressions are identical, implying uniqueness of prime factorization.
    }%
}
\newpage   
    \item {\bf Chinese remainder theorem.} Suppose $gcd(a,b)=1$. Prove that given any integer $r$ and any integer $s$, there exists an integer $n$ such that $a|n-r$ and $b|n-s$. How does one explicitly find such $n$? To what extent is it unique?

 \fbox{%
    \parbox{\dimexpr\linewidth-2\fboxsep}{%
\textbf{Solution (viii):}

Since, $gcd(a,b) = 1$, we can find $x,y \in \mathbb{Z}$ such that, $ax + by =  1$.

\smallskip

Now, consider $n = axs + byr$.

\smallskip

Then $n-r = axs + r(by-1) = axs - axr = ax(s-r)$. Hence, $a | n-r$. Similarly $n-s = s(ax-1) + byr = byr - bys = by(r-s)$. Hence, $b | r-s$.

\smallskip

So, in order to explicitly find such a $n$, we have to first find the values of $x$ and $y$, satisfying $ax + by = 1$, which we can find from \textbf{(iv)}.

\smallskip

Now, for the uniqueness for $n$, as $x$ and $y$, such that $ax + by = 1$, are not unique (again refer to \textbf{(iv)}), for different combination of $x$ and $y$, we can find different $n$, satisfying these properties. I will, explicitly show this now.

\smallskip

Suppose we found a pair of $x$ and $y$, using \textbf{(iv)}. Now, as shown in $(iv)$, $x' = x + kb$ and $y' = y - ka$ for $k \in \mathbb{Z}$, also satisfy $ax' + by' = 1$. Hence, we can find a new $n'$, using $x'$ and $y'$- $$n' = asx' + bry' = as(x + kb) + br(y - ka) = (asx + bry) + abks - abkr = n + abk(s-r)$$
Hence, given we find one $n$, we can find many others, using this expression.
    }%
}
\bigskip 
    \item {\bf Example.} Carry out parts (iii) and (iv) for $(a,b) = ($your roll number of form 201xxx, 2017). Then do part (viii) for $(r,s) = (20,19)$.

 \fbox{%
    \parbox{\dimexpr\linewidth-2\fboxsep}{%
\textbf{Solution (ix):}

First, I find $gcd(201944,2017)$ using \textit{Euclidean Algorithm}, as portrayed in \textbf{(iii)}.

$$201944 = 100.2017 + 244$$
$$2017 = 8.244 + 65$$
$$244 = 3.65 + 49$$
$$65 = 1.49 + 16$$
$$49 = 3.16 + 1$$
$$16 = 16.1 + 0$$

Hence, $gcd(201944,2017) = 1$.

\bigskip

Now, I use these above equations to reverse engineer a $s$ and $t$ in $\mathbb{Z}$, such that $201944s + 2017t = 1$.

$$1 = 1.49 - 3.16$$
$$= 1.49 - 3.(65-1.49) = 4.49 - 3.65$$
$$= 4.(244-3.65) - 3.65 = 4.244 - 15.65$$
$$= 4.244 - 15.(2017 - 8.244) = 124.244 - 15.2017$$
$$= 124.(201944-100.2017) - 15.2017 = 124.201944 - 12415.2017$$

Hence $gcd(201944,2017) = 1 = 201944(124) + 2017(-12415)$. Here, $s = 124$ and $t = -12415$.

\bigskip

Now, I find a $n$ such that $201944|n-20$ and $2017|n-19$. As, shown in \textbf{(viii)}, $n = 201944.19.124 + 2017.20.(-12415) = -25041036$. Hence, we are done.
    }%
}

\end{enumerate}

\end{document}