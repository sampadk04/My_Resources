\documentclass[12pt,a4paper]{article}
\usepackage[utf8]{inputenc}
\usepackage{graphicx}
\usepackage{graphics}
\graphicspath{./images/}
\usepackage{url}
\usepackage{amsmath,amsfonts,amsthm,amssymb,color,amsbsy}
\usepackage[pagebackref=true,colorlinks]{hyperref}
\usepackage{tikz,pgf}
\usepackage{tikz-cd}
\usepackage{mathrsfs}
\usepackage{xlop}
\usepackage{soul}
\usepackage{xcolor}
\graphicspath{ {./Images/} }
\newtheorem{theorem}{Theorem}[section]
\newtheorem{corollary}[theorem]{Corollary}
\newtheorem{lemma}[theorem]{Lemma}

\addtolength{\oddsidemargin}{-.75in}
\addtolength{\evensidemargin}{-.75in}
\addtolength{\textwidth}{1.75in}

\addtolength{\topmargin}{-.875in}
\addtolength{\textheight}{1.75in}
\hypersetup{
	colorlinks=true,
	linkcolor=blue,
	citecolor=magenta
}


\definecolor{xdxdff}{rgb}{0.49019607843137253,0.49019607843137253,1.}
\definecolor{zzttqq}{rgb}{0.6,0.2,0.}
\definecolor{qqqqff}{rgb}{0.,0.,1.}

\newtheorem{proposition}[theorem]{Proposition}
\newtheorem{conjecture}[theorem]{Conjecture}
\newtheorem{claim}[theorem]{Claim}
\newtheorem{fact}[theorem]{Fact}
\newtheorem{assumption}[theorem]{Assumption}
\newtheorem{warning}[theorem]{Warning}

\theoremstyle{definition}
\newtheorem{definition}[theorem]{Definition}
\newtheorem{example}[theorem]{Example}
\newtheorem{remark}[theorem]{Remark}
\newtheorem{exercise}[theorem]{Exercise}
\newtheorem{recall}[theorem]{Recall}
\newtheorem{observation}[theorem]{Observation}

\newcommand\numberthis{\addtocounter{equation}{1}\tag{\theequation}}
\newcommand{\F}{\mathbb{F}}
\newcommand{\Z}{\mathbb{Z}}
\newcommand{\N}{\mathbb{N}}
\newcommand{\Q}{\mathbb{Q}}
\newcommand{\R}{\mathbb{R}}
\newcommand{\C}{\mathbb{C}}
\newcommand{\K}{\mathbb{K}}
\newcommand{\p}{\mathbb{P}}
\newcommand{\e}{\epsilon}
\newcommand{\A}{\mathcal{A}}
\newcommand{\V}{\mathbf{v}}
\newcommand{\X}{\mathcal{X}}
\newcommand{\n}{\mathcal{N}}
\DeclareMathOperator{\im}{im}
\DeclareMathOperator{\sgn}{sgn}
\DeclareMathOperator{\tr}{tr}
\DeclareMathOperator{\adj}{adj}
\DeclareMathOperator{\real}{Re}
\DeclareMathOperator{\imag}{Im}
\newcommand{\bigO}{\mathcal{O}}
\newcommand{\ds}{\displaystyle}
\newcommand{\bs}{\boldsymbol}
\newcommand{\cl}{\overline}
\newcommand{\inr}[1]{\left\langle #1 \right\rangle}
\newcommand{\nrm}[1]{\left\| #1 \right\|}
\newcommand{\abs}[1]{\left| #1 \right|}
\newcommand{\set}[1]{\left{ #1 \right}}
\newcommand{\mat}[1]{\begin{bmatrix}#1\end{bmatrix}}

\title{Algebra III - Homework 2}
\author{Sampad Kumar Kar -- BMC201944}
\date{\today}


\begin{document}

\maketitle

\begin{flushleft}

\setcounter{section}{10}
\section{Problem}

{\bf Artin: 11.4.3} Identify the following rings:

(b) $\mathbb{Z}[i]/(2+i)$

(e) $\mathbb{Z}[x]/(x^2 +3,5)$

\medskip

(Here take ``identify" to mean ``find cardinality and whether the ring is a field/integral domain" You can say more if you like.)

6
{\bf Solution 11}

\medskip

{\bf Artin: 11.4.3 (b)}

We know that $\mathbb{Z}[i] \cong \mathbb{Z}[x]/(x^2+1) $. By {\it Correspondence Theorem}, $\mathbb{Z}[i]/(i+2) \cong \mathbb{Z}[x]/(x^2+1,x+2) $ (as $(i+2)$ in $\mathbb{Z}[i]$ corresponds to the ideal $(x^2+1,x+2)$ in $\mathbb{Z}[x]$).

Also $\mathbb{Z}[x]/(x+2) \cong \mathbb{Z}$ via the map $x \mapsto -2$. Now, the ideal in $\mathbb{Z}$ corresponding to the ideal $(x+2,x^2+1)$ under this map is the principal ideal generated by $2^2 + 1 = 5$. So, again by {\it Correspondence Theorem} $\mathbb{Z}[x]/(x+2,x^2+1) \cong \mathbb{Z}/(5) = \mathbb{Z}/5\mathbb{Z} \cong \mathbb{F}_5$.

\medskip

So, $\mathbb{Z}[i]/(2+i)$ is a field (hence domain) with cardinality $5$.

\medskip

{\bf Artin: 11.4.3 (e)}

Firstly, $\mathbb{Z}[x]/(5) \cong \mathbb{F}_5[x]$ (we just take coefficient modulo $5$ and reduce it). Again by {\it Correspondence Theorem}, we just have to consider $\mathbb{F}_5[x]/(x^2+3)$ which is the ring $R$ obtained by adjoining an element $\alpha$ to $\mathbb{F}$ by the relation $\alpha^2 +3 = 0$ By {\bf Proposition 11.5.7} of Artin (pg - 339), elements of $R$ are generated by $(1,\alpha)$ and hence are linear expressions $a + b\alpha$ with coefficients in $\mathbb{F}_5$.

Now, $R$ is a field of order $25$, because every non-zero element $a + b\alpha$ of $R$ is invertible. This is because - $c = (a+b\alpha)(a-b\alpha) = a^2 + 3b^2$. This is an element of $\mathbb{F}_5$, and because $3$ isn't  square in $\mathbb{F}_5$, $c \ne 0$ unless both $a = b = 0$. So, if $a + b\alpha \ne 0$, $c \ne 0$ is invertible in $\mathbb{F}$ and hence inverse of $a + b\alpha$ is $(a - b\alpha)c^{-1}$.

\medskip

So, $\mathbb{Z}[x]/(x^2+3,5)$ is a field with cardinality $25$.

\newpage

\setcounter{section}{11}
\section{Problem}

{\bf Artin: 11.5.3} Describe the ring obtained from $\mathbb{Z}/12\mathbb{Z}$ by adjoining an inverse of $2$.

\bigskip

{\bf Solution 12}

\medskip

Let $\mathbb{Z}/12\mathbb{Z} = \mathbb{Z}_{12}$. We want the ring obtained by adjoining $\mathbb{Z}_{12}$ with inverse of $2$, which is isomorphic to $\mathbb{Z}_{12}[x]/(2x-1)$.

\medskip

Firstly, I will show that $(2x-1) = (3,x+1)$.

Clearly, $2(x+1) - 3 = 2x-1 \implies (2x-1) \subseteq (3,x+1)$. 

Now, $3 = -3(2x-1) -6x(2x-1)$ and $x+1 = 3x - (2x-1) = -3x(2x-1) - 6x^2(2x-1) - (2x-1)$, which means $(3,x+1) \subseteq (2x-1)$.

\medskip

This means $\mathbb{Z}_{12}/(2x-1) \cong \mathbb{Z}_{12}/(3,x+1) \cong \mathbb{F}_{3}[x]/(x+1)$ as $\mathbb{Z}_{12}[x]/(3) \cong \mathbb{F}_{3}[x]$ (just by reducing the coefficients again) and $\mathbb{F}_{3}[x]/(x+1) \cong \mathbb{F}_{3}$ via the map $x \mapsto -1$.

\medskip

Hence $\mathbb{Z}_{12}/(2x-1) \cong \mathbb{F}_{3}$.

\newpage

\setcounter{section}{12}
\section{Problem}

{\bf Artin: 11.5.4} Determine the structure of the ring $R'$ obtained from $\mathbb{Z}$ by adjoining an element $\alpha$ satisfying each set of relations.

(a) $2\alpha = 6$, $6\alpha = 15$,

(b) $2\alpha - 6 = 0$, $\alpha - 10 = 0$

\bigskip

{\bf Solution 13}

\medskip

{\bf Artin: 11.5.4 (a)} The ring obtained is $\mathbb{Z}[x](2x-6,6x-15)$.

Clearly $(2x-6,6x-15) = (2x-6,3)$ because $3 = (6x-15) - 3(2x-6)$ and $6x-15 = 3(2x-6) + 3$.

This means, $\mathbb{Z}[x]/(2x-6,6x-15) = \mathbb{Z}[x]/(2x-6,3) \cong \mathbb{F}_3[x]/(2x-6) \cong \mathbb{F}_3[x]/(-x) \cong \mathbb{F}_3$ via the map $x \mapsto 0$ (as $(2x-6,3)$ in $\mathbb{Z}[x]$ corresponds to the ideal $(2x-6) = (-x)$ in $\mathbb{F}_3[x]$).

\medskip

{\bf Artin: 11.5.4 (b)} The ring obtained is $\mathbb{Z}[x]/(2x-6,x-10)$.

Clearly $(2x-6,x-10) = (14,x-10)$ because $14 = (2x-6) - 2(x-10) $ and $2x-6 = 14 + 2(x-10)$.

This means $\mathbb{Z}[x]/(2x-6,x-10) = \mathbb{Z}[x]/(14,x-10) \cong \mathbb{Z}_{14}[x]/(x-10) \cong \mathbb{Z}_{14} = \mathbb{Z}/14\mathbb{Z}$ via the map $x \mapsto 10$.

\newpage

\setcounter{section}{13}
\section{Problem}

{\bf Artin: 11.5.5} Are there fields $F$ such that the rings $F[x]/(x^2)$ and $F[x]/(x^2-1)$ are isomorphic?

\medskip

(Hint: consider maximal ideals.)

\bigskip

{\bf Solution 14}

\medskip

\begin{claim}
	For a field $\mathbb{F}$, the rings $\mathbb{F}/(x^2) \cong \mathbb{F}/(x^2-1)$ iff $\mathbb{F}$ has characteristic $2$, i.e. $1 + 1 = 0$ or $1 = -1$ in $\mathbb{F}$.
\end{claim}

\begin{proof}
	Suppose $\mathbb{F}$ has characteristic $2$. Consider the map given by $x \mapsto x+1$. By {\it Correspondence Theorem}, the ideal $(x^2)$ corresponds to $((x+1)^2) = (x^2 - 1)$ (as $(x+1)^2 = x^2 + (1+1)x + 1$ $ = x^2 +1 = $ $ x^2 -1$). Since, all the hypothesis of of the theorem are satisfied, $\mathbb{F}[x]/(x^2) \cong \mathbb{F}[x]/(x^2-1)$.

	\medskip

	Coversely, suppose $\mathbb{F}$ is not of characteristic $2$, i.e. $1 \ne -1$ or $1 + 1 \ne 0$.

	Now, in $\mathbb{F}[x]/(x^2)$, and consider the coset of $x$, say $\overline{x}$. Then clearly $\overline{x}^2 = 0$, meaning $\overline{x}$ is nilpotent with degree $2$, but there is no such element in $\mathbb{F}/(x^2-1)$ because if there exists some $ax+b \ne 0 \in \mathbb{F}$ such that $(ax+b)^2 = f(x)(x^2-1)$ for some $f \in \mathbb{F}[x]$, then $deg(f) = 0$. This means $(ax + b)^2 = c(x^2 -1) \implies 2ab = 0$, which is not possible since, $\mathbb{F}$ is a field and $ax + b \ne 0$.
\end{proof}

\newpage

\setcounter{section}{14}
\section{Problem}

{\bf Artin: 11.8.2} Determine the maximal ideals of each of the following rings, and also identify which of the given rings are fields:

(b) $\mathbb{R}[x]/(x^2)$,

(c) $\mathbb{R}[x]/(x^2-3x+2)$,

(d) $\mathbb{R}[x]/(x^2+x+1)$.

\bigskip

{\bf Solution 15} 

\medskip

{\bf Artin: 11.8.2 (b)} Now, $(x)$ is the only proper ideal in $\mathbb{R}[x]$ that contains the ideal $(x^2)$, by {\it Correspondence Theorem}, $(\overline{x})$ is only proper ideal, and hence the maximal ideal in $\mathbb{R}[x]/(x^2)$ which means $\mathbb{R}[x]/(x^2)$ is not a field.

\medskip

{\bf Artin: 11.8.2 (c)} Now, $(x-1)$ and $(x-2)$ are the only ideals in $\mathbb{R}[x]$ containing the ideal $(x^2-3x+2)$ and again by {\it Correspondence Theorem}, $(\overline{x-1})$ and $(\overline{x-2})$ are the only proper ideals and hence the maximal ideals of $\mathbb{R}[x]/(x^2 - 3x +2)$, which also means $\mathbb{R}[x]/(x^2 - 3x +2)$ is not a field.

\medskip

{\bf Artin: 11.8.2 (d)} We can can show that $(x^2 +x +1)$ is a maximal ideal in $\mathbb{R}[x]$. Suppose there was an ideal $I$ containing $(x^2 + x +1)$. Since $\mathbb{R}$ is a field, $I$ is pricipal. Say $I = (f)$ for some $f \in \mathbb{R}[x]$. Clearly $f(x) | x^2+x+1$. But $x^2 + x + 1$ isn't factorisable non-trivialy in $\mathbb{R}$, which means there are no proper ideals containing $(x^2 + x + 1)$. Again by {\it Correspondence Theorem}, there are no proper ideals in $\mathbb{R}[x]/(x^2 + X +1)$, which means it is a field.

\newpage

\setcounter{section}{15}
\section{Problem}

Suppose you are {\it given} a finite field $E$. Show that $ |E| = p^n$ for a prime number $p$. (Hints: (i) First identify $p$ from $F$ using the first isomorphism theorem. Which other ring should you use? (ii) If a field $F$ is subfield of a ring $E$, then note that $E$ is in particular a vector space over $F$ with the given operations. We will use this repeatedly, especially when $E$ is a field as well. In particular we can consider dimension of $E$ over $F$, which we call the degree of $E$ over $F$, denoted $[E:F]$.)

\medskip

Keep a list of known facts for finite fields. So far you know (1)  that cardinality is always = $p^n$  for some prime $p$, (2) the Frobenius map from exercise 11.3.8, (3) that each element of such a field of cardinality $q$ (if such a field exists) is a root of $x^q - x$. Add to this list every time you see something new that can be applied and keep trying to show existence and uniqueness of a field of cardinality $p^n?$ Can you do it now?! %(Need more - existence: attaching roots! uniqueness: cyclicity of F cross)

\bigskip

{\bf Solution 16}

\medskip

First, consider the characteristic map from $\mathbb{Z}$ to the field $E$. Let $F$ be the image of this map in $E$. Let $K$ be the kernel of this map. We know that any ideal in $\mathbb{Z}$ is a principal ideal. So, $K = (n) = n\mathbb{Z}$ for some $n \in \mathbb{Z}$.

\medskip

Now, $F$ is a subring of $E$ and since $E$ is a domain, $F$ is a also a domain. And by the {\it First Isomorphism Theorem}, $F \cong \mathbb{Z}/n\mathbb{Z}$. But this is only possible if $n = p$ for some prime (else $\mathbb{Z}/n\mathbb{Z}$ won't be a domain). Thus $F$ is a field of cardinality $p$.

\medskip

Now, $E$ forms a finite dimensional vector space over $F$. Suppose its dimension is $m$. Then $E \cong F^m$, which means cardinality of $E$ is $p^m$ for some prime $p$.

\newpage

\end{flushleft}
\end{document}